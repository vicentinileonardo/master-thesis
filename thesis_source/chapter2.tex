\chapter{Background}
\label{cha:background}

\section{GreenOps landscape}


from greenops landscape itself

In the context of cloud-native sustainability,
the Technical Advisory Group (TAG) Environmental Sustainability is a XXX that supports and advocates for environmental sustainability initiatives in cloud native technologies.


green software foundation

proposed a standard for data like the FOCUS standard available 
as per 2025 this standard is not yet adopted by cloud providers

developed the Impact Framework which will be described in section XY

\subsubsection{Green Software foundation}

trying to push a specification similar to what focus is for FinOps



\section{Cloud providers}

cloud regions


regions vs availability zones
Cloud providers usually further divide region into ...


\subsection{Multi cloud}
(why, how to achieve)

advantages

reduces vendor lock-in


\subsection{computational sustainability by cloud providers}

\section{Kubernetes}


\subsection{Kubernetes as a platform}
Kubernetes as a platform to manage things


Many cloud-native development teams work with a mix of configuration systems, APIs, and tools to manage their infrastructure. This mix is often difficult to understand, leading to reduced velocity and expensive mistakes. Config Connector provides a method to configure many Google Cloud services and resources using Kubernetes tooling and APIs.
%(https://cloud.google.com/config-connector/docs/overview)

%https://cloud.google.com/config-connector/docs/concepts/resources#managing_resources_with_kubernetes_objects

\subsection{Kubernetes extendability}

\subsubsection{Operator paradigm}

CRDs


\section{Krateo}
\label{sec:krateo}

what is krateo. Recognized by Gartner
by 2025 companies without a ... (cite)

architecture, components
core provider, cdc
helm charts as native resources

values.schema.json



\section{State of the Art}
An extensive analysis of existing systems have been made in order to...

\subsection{CASPER}

CASPER (Carbon-Aware Scheduling and Provisioning for Distributed Web Services) is a carbon-aware scheduling and provisioning system whose primary purpose is to minimize the carbon footprint of distributed web services \cite{Souza_2023}.
The system is defined as a multi-objective optimization problem that considers two factors: the \textbf{variable carbon intensity} and the \textbf{latency constraints} of the network \cite{Souza_2023}.
By evaluating the framework in real-world scenarios, the authors demonstrate that CASPER achieves significant reductions in carbon emissions (up to 70\%) while meeting application \textbf{Service Level Objectives (SLOs)}, highlighting its potential for practical implementation in large-scale distributed systems \cite{Souza_2023}. However, the system CANNOT BE CONSIDERED A REAL PRODUCTION SYSTEM.




\subsection{CASPIAN}

most important ptobably

\subsection{Let'sWaitAwhile}

test

\subsection{Other systems}

carbonScaler




\subsection{SOTA Recap}

many simulation, no real system
no much flexibility






\newpage

