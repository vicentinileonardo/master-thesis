\chapter{Conclusion}
\label{cha:conclusion}


production-ready system

\section{Future improvements}

day2 operations
we are ready for this

Scaling down a VM (example of Day 2 operations)
From: (4 vCPU, 8 GiB RAM)
To: (2 vCPU, 4 GiB RAM)

This use case is meaningful for workloads with durations in the order of at least days. Otherwise, for short-lived workload this use case does not make sense.
And in the case of workloads with days as duration, time and geographical shifting is not that relevant.

This use case will leverage system and performance metrics.








support for other resources
we need
templates
operators



there is one paper (https://ceur-ws.org/Vol-2382/ICT4S2019_paper_28.pdf) that uses local air temperature and solar irradiance as tiebreaker for 2 datacenters with similar carbon intense grid. 
“Local air temperature surrounding a datacentre affects the amount of energy needed for cooling”. 
They also claim that: “Solar irradiance varies more widely than carbon intensity across global regions”.

Maybe it could be an extension of our system in the future.


\subsection{Multi model serving}
"The original design of KServe deploys one model per InferenceService. But, when dealing with a large number of models, its 'one model, one server' paradigm presents challenges for a Kubernetes cluster."

kserve model mesh instead of several InferenceService
there is a lot of overhead in the current configuration

how much is better to use more models instead of one generic model 




\newpage

