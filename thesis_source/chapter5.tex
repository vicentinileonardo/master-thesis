\chapter{Discussion}
\label{cha:discussion}

\section{End-to-end integrated test}

final result
A comprehensive end-to-end integrated test has been carried out on a Kubernetes cluster

this was used to validate the system

%(dependency graph)

\section{GreenOps system evaluation}

\subsection{Theoretic upper bound}

 (how close can we get, masachussets amherest group)

\subsection{Baseline definition}

We should prepare one or more baseline schedulings that will be used as a baseline and compared with a carbon-aware scheduling proposed by our system.

\subsection{Black hole phenomenon}

How to deal with the so-called “Black hole” phenomenon?
That is, if 100 workload scheduling arrives at some point, there is the possibility that the outcome of the system we are building is: “schedule all workloads in Norway” where Norway is the region with least carbon intensity at that moment.
This phenomenon came up also in a previous meeting but it is not clear if this could be a problem etc..
A probable differentiator could be the max latency field of the workload request. Other service requirements could contribute to this as well.


(how it is countered)

\subsection{Side effects}

Maybe out of scope of this work, side effects, big picture.
What happens if a big percentage of companies that relies on cloud services starts to adopt carbon-aware scheduling of their workloads?
We tend to image cloud providers or even cloud regions as an infinite pool of resources, and at a certain level it is almost like that. But could carbon aware scheduling have larger, not foreseen, side effects?
Is this a responsibility of who schedules? Shall schedulers be responsible for the load on regions? Like self-imposing some sort of limits/caps. Ethics?

\subsection{Preliminary evaluation}

\begin{comment}
for the purpose of this theses

boavizta API simulation

assumptions
- analysis limited to only cloud VM, (aligned with the scope of this theses)
- data related to GCP is not data from boavizta (even if gcp is supported in our current system) but mapped from azure and aws

limitations
- whole countries, not regions


not easily integratable in a real production system due to its quite restrictive license (AGPL 3)
it is still usable for research purposes like in this case.
\end{comment}






\newpage
